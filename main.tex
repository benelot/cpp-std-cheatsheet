\documentclass[11pt,article,oneside,a4paper]{memoir}

%% Packages
%% ========

%% many common packages
\input{commonpackages}

%% Some more packages that you may want to use.  Have a look at the
%% file, and consult the package docs for each.
\input{extrapackages}

%% Our layout configuration.
\input{layoutsetup}

%% Theorem environments.  You will have to adapt this for a German
%% thesis.
\input{theoremsetup}

%% Helpful macros.
\input{macrosetup}

%%page layout settings and listing templates etc.
\input{settings}

\usepackage{listings}
\usepackage{color}
 
\definecolor{codegreen}{rgb}{0,0.6,0}
\definecolor{codegray}{rgb}{0.5,0.5,0.5}
\definecolor{codepurple}{rgb}{0.58,0,0.82}
\definecolor{backcolour}{rgb}{0.95,0.95,0.92}
 
\lstdefinestyle{mystyle}{
    backgroundcolor=\color{backcolour},   
    commentstyle=\color{codegreen},
    keywordstyle=\color{magenta},
    numberstyle=\tiny\color{codegray},
    stringstyle=\color{codepurple},
    basicstyle=\footnotesize,
    breakatwhitespace=false,         
    breaklines=true,                 
    captionpos=b,                    
    keepspaces=true,                 
    numbers=left,                    
    numbersep=5pt,                  
    showspaces=false,                
    showstringspaces=false,
    showtabs=false,                  
    tabsize=2
}
 
\lstset{style=mystyle}

\title{CPP std::Cheatsheet}
\author{
Benjamin Ellenberger\\
\vspace{2em}
Github (git/svn) repository page:\\ \url{https://github.com/benelot/cpp-std-cheatsheet}\\
Contact \href{mailto:be.ellenberger@gmail.com}{be.ellenberger@gmail.com} if you have any questions.}
\thesistype{Preparation for a Coding Interview}
%\advisors{}
\department{Institute of Neuroinformatics}
\date{\today}

\begin{document}
\frontmatter

%% Title page is auto-generated from document information above.
%% DO NOT CHANGE.
\begin{titlingpage}
  \calccentering{\unitlength}
  \begin{adjustwidth*}{\unitlength-24pt}{-\unitlength-24pt}
    \maketitle
  \end{adjustwidth*}
\end{titlingpage}

\mainmatter
\newpage
\chapterprecishere{---}
\newpage

%% This change is needed if the article option for the memoir document class
%% is used, in order to count sections (article) as if they were chapters (memoir)
\counterwithout{section}{chapter}

%% Our content

\newpage
\clearpage
\pagenumbering{roman}
\setcounter{tocdepth}{3}
\setcounter{secnumdepth}{2}
\tableofcontents



\clearpage
\pagenumbering{arabic}

\newpage

\section{Data structures}
\todo[inline]{Describe them.}
\subfile{Data-structure-String.tex}
\subfile{Data-structure-ArrayList.tex}
\subfile{Data-structure-Binary-Search-Tree.tex}
\subfile{Data-structure-Map.tex}
\subfile{Data-structure-Unordered-Map.tex}
\subfile{Data-structure-Linked-List.tex}
\subfile{Data-structure-Queue.tex}
\subfile{Data-structure-Set.tex}
\subfile{Data-structure-Unordered-Set.tex}
\subfile{Data-structure-Stack.tex}
\subfile{Data-structure-Vector.tex}

\section{Algorithms}

\todo[inline]{Describe them.}
\subsection{Graph Algorithms}
\subfile{Algorithm-BFS}
\subfile{Algorithm-DFS}
\subfile{Algorithm-Dijkstra}

\subsection{Sorting Algorithms}
\subfile{Algorithm-Heap-Sort}
\subfile{Algorithm-Merge-Sort}
\subfile{Algorithm-Quick-Sort}

\subsection{Searching Algorithms}
\subfile{Algorithm-Binary-Search}
%\subfile{Algorithm-Template.tex}

\section{Concepts}
\todo[inline]{Describe them.}
\subfile{Concept-Binary-Manipulation}
\subfile{Concept-Memory}
\subfile{Concept-Recursion}
\subfile{Concept-Dynamic-Programming}
\subfile{Concept-Big-O-Time-Space}
%\subfile{Concept-Template}

%\subfile{13-LearningTimeSeries.tex}

%\pagebreak
%\def\cheatsheet{2014}
%\subfile{15-Cheatsheet2014-BE.tex}

\subfile{Glossary.tex}

\subfile{TODO.tex}

\end{document}
