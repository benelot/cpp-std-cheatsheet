\documentclass[main]{subfiles}
\begin{document}

%@@@@@@@@@@@@@@@@@@@@@@@@@@@@@@
% summarizes lecture 
% author:

\subsubsection{Merge Sort}
\renewcommand{\arraystretch}{1.5}
\definecolor{lgray}{gray}{0.95}
\definecolor{gray}{gray}{0.9}

\rowcolors{1}{lgray}{gray}

Conceptually, a merge sort works as follows:

\begin{enumerate}
\item Divide the unsorted list into n sublists, each containing 1 element (a list of 1 element is considered sorted). (Time: O(n), Space: O(n) total)
\item Repeatedly merge sublists to produce new sorted sublists until there is only 1 sublist remaining. This will be the sorted list. (Time: O(n log(n), Space: O(n) total, O(n) auxiliary)
\end{enumerate}

\begin{tabular}{ll}
Worst case performance & O(n log n)\\
Best case performance & O(n log n) typical,O(n) natural variant\\
Average case performance & O(n log n)\\
Worst case space complexity & O(n) total, O(n) auxiliary\\
\end{tabular}

\scriptsize
\begin{lstlisting}[language=C++]
//! \brief Performs a recursive merge sort on the given vector
//! \param vec The vector to be sorted using the merge sort
//! \return The sorted resultant vector after merge sort is
//! complete.
vector<int> merge_sort(vector<int>& vec)
{
    // Termination condition: List is completely sorted if it
    // only contains a single element.
    if(vec.size() == 1)
    {
        return vec;
    }

    // Determine the location of the middle element in the vector
    std::vector<int>::iterator middle = vec.begin() + (vec.size() / 2);

	// Split vector
    vector<int> left(vec.begin(), middle);
    vector<int> right(middle, vec.end());

    // Perform a merge sort on the two smaller vectors
    left = merge_sort(left);
    right = merge_sort(right);

    return merge(vec,left, right);
}

//! \brief Merges two sorted vectors into one sorted vector
//! \param left A sorted vector of integers
//! \param right A sorted vector of integers
//! \return A sorted vector that is the result of merging two sorted
//! vectors.
vector<int> merge(vector<int> &vec,const vector<int>& left, const vector<int>& right)
{
    // Fill the resultant vector with sorted results from both vectors
    vector<int> result;
    unsigned left_it = 0, right_it = 0;

    while(left_it < left.size() && right_it < right.size())
    {
        // If the left value is smaller than the right it goes next
        // into the resultant vector
        if(left[left_it] < right[right_it])
        {
            result.push_back(left[left_it]);
            left_it++;
        }
        else
        {
            result.push_back(right[right_it]);
            right_it++;
        }
    }

    // Push the remaining data from both vectors onto the resultant
    while(left_it < left.size())
    {
        result.push_back(left[left_it]);
        left_it++;
    }

    while(right_it < right.size())
    {
        result.push_back(right[right_it]);
        right_it++;
    }
    
    //show merge process..
    int i;
    for(i=0;i<result.size();i++)
    {                                
	   cout <<result[i] <<" ";
    }
    
    // break each line for display purposes..
        cout<<"***********"<<endl; 

    //take a source vector and parse the result to it. then return it.  
	vec = result;				
	return vec;
}
\end{lstlisting}


\end{document}