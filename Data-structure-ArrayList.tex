\documentclass[main]{subfiles}
\begin{document}

%@@@@@@@@@@@@@@@@@@@@@@@@@@@@@@
% summarizes lecture 
% author:

\subsection{Array List}
\renewcommand{\arraystretch}{1.5}
\definecolor{lgray}{gray}{0.95}
\definecolor{gray}{gray}{0.9}

\rowcolors{1}{lgray}{gray}

Some introduction text.

\scriptsize
\begin{longtable}{p{0.4\linewidth} p{0.6\linewidth}}
\hline \textbf{Method} & \textbf{Description}\\ \hline
\endfirsthead

\hline \textbf{Method} & \textbf{Description}\\ \hline\hline
\endhead
boolean \textbf{add}(E e) &
Appends the specified element to the end of this list.\\

void \textbf{add}(\textit{int} index, \textit{E} element) &
Inserts the specified element at the specified position in this list.\\

boolean \textbf{addAll}(\textit{Collection}$<?~extends~E>$ c) &
Appends all of the elements in the specified collection to the end of this list, in the order that they are returned by the specified collection's Iterator.\\
boolean \textbf{addAll}(\textit{int} index, Collection$<?~extends~E>$ c) &
Inserts all of the elements in the specified collection into this list, starting at the specified position.\\

void \textbf{clear}() &
Removes all of the elements from this list.\\

Object \textbf{clone}() &
Returns a shallow copy of this ArrayList instance.\\

boolean \textbf{contains}(\textit{Object} o) &
Returns true if this list contains the specified element.\\

void \textbf{ensureCapacity}(\textit{int} minCapacity) &
Increases the capacity of this ArrayList instance, if necessary, to ensure that it can hold at least the number of elements specified by the minimum capacity argument.\\

E \textbf{get}(\textbf{int} index) &
Returns the element at the specified position in this list.\\

int \textbf{indexOf}(\textit{Object} o) &
Returns the index of the first occurrence of the specified element in this list, or -1 if this list does not contain the element.\\
boolean \textbf{isEmpty}() &
Returns true if this list contains no elements.\\
Iterator$<E>$ iterator() &
Returns an iterator over the elements in this list in proper sequence.\\
int \textbf{lastIndexOf}(\textit{Object} o) &
Returns the index of the last occurrence of the specified element in this list, or -1 if this list does not contain the element.\\
\textbf{ListIterator}$<E>$ \textbf{listIterator}() &
Returns a list iterator over the elements in this list (in proper sequence).\\
\textbf{ListIterator}$<E>$ \textbf{listIterator}(\textit{int} index) &
Returns a list iterator over the elements in this list (in proper sequence), starting at the specified position in the list.\\
E \textbf{remove}(int index) &
Removes the element at the specified position in this list.\\
boolean \textbf{remove}(Object o) &
Removes the first occurrence of the specified element from this list, if it is present.\\
boolean \textbf{removeAll}(Collection$<?>$ c) &
Removes from this list all of its elements that are contained in the specified collection.\\
protected void removeRange(\textit{int} fromIndex, \textit{int} toIndex) &
Removes from this list all of the elements whose index is between fromIndex, inclusive, and toIndex, exclusive.\\
boolean \textbf{retainAll}(Collection$<?>$ c) &
Retains only the elements in this list that are contained in the specified collection.\\
E set(int index, E element) &
Replaces the element at the specified position in this list with the specified element.\\
int size() &
Returns the number of elements in this list.\\
List$<E>$ \textbf{subList}(\textit{int} fromIndex, \textit{int} toIndex) &
Returns a view of the portion of this list between the specified fromIndex, inclusive, and toIndex, exclusive.\\
Object[] \textbf{toArray}() &
Returns an array containing all of the elements in this list in proper sequence (from first to last element).\\
$<T>$ $T[]$ \textbf{toArray}(\textit{T[]} a) &
Returns an array containing all of the elements in this list in proper sequence (from first to last element); the runtime type of the returned array is that of the specified array.\\
void \textbf{trimToSize}() &
Trims the capacity of this ArrayList instance to be the list's current size.\\
\end{longtable}
\todo[inline]{}
\end{document}